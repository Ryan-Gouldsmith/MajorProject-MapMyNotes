\documentclass[11pt,fleqn,twoside]{article}
\usepackage{makeidx}
\makeindex
\usepackage{palatino} %or {times} etc
\usepackage{plain} %bibliography style
\usepackage{amsmath} %math fonts - just in case
\usepackage{amsfonts} %math fonts
\usepackage{amssymb} %math fonts
\usepackage{lastpage} %for footer page numbers
\usepackage{fancyhdr} %header and footer package
\usepackage{mmpv2}
\usepackage{url}
\usepackage[subtle]{savetrees}


% the following packages are used for citations - You only need to include one.
%
% Use the cite package if you are using the numeric style (e.g. IEEEannot).
% Use the natbib package if you are using the author-date style (e.g. authordate2annot).
% Only use one of these and comment out the other one.
\usepackage{cite}
%\usepackage{natbib}

\begin{document}

\name{Ryan Gouldsmith}
\userid{ryg1}
\projecttitle{MapMyNotes}
\projecttitlememoir{MapMyNotes} %same as the project title or abridged version for page header
\reporttitle{Outline Project Specification}
\version{0.4}
\docstatus{Draft}
\modulecode{CS39440}
\degreeschemecode{G401}
\degreeschemename{Computer Science}
\supervisor{Hannah Dee} % e.g. Neil Taylor
\supervisorid{hmd}
\wordcount{}

%optional - comment out next line to use current date for the document
%\documentdate{10th February 2014}
\mmp

\setcounter{tocdepth}{3} %set required number of level in table of contents


%==============================================================================
\section{Project description}
%==============================================================================
MapMyNotes aims to produce a web application which will allow the user to upload an image of their handwritten notes and store them in a database. It should also provide the functionality to search for a note and add them to a calendar.

People still handwrite their notes and this often leads to large amounts of paper. This can be cumbersome especially if they try to search through notes which may not be organised. There are modern applications which will allow you to take notes, such as EverNote [CITE], but people still like writing handwritten notes.

At a basic level the application will allow the user to upload an image of their handwritten notes. It will use very basic OCR analysis to identify the title of the notes, the lecturer, the module code, the location and time. Additionally, at the basic level it will interpret the author's handwriting.

From this the application can manually tag the module code for the notes based on suggestions from the OCR analysis, but ultimately the user tags their own notes. The notes metadata (code, location, lecturer and time) will be stored in the database with the note itself. The user can then search for a given module code and find all associated notes, for the specific user. Finally, the user must be able to insert the notes into a calendar, for archival purposes; OAuth with Google Calendar has been considered but more research needs to be conducted.

If there's sufficient time remaining then the application will look to do full OCR recognition on the note and produce a document which the note has been converted to text. There will also be an extraction of diagrams and graph from the note and the user will be able to drag, rotate etc. Additionally, another further aim is to automatically rotate the image of the note, allowing for notes to be taken at obscure angles and the system will help this. Finally, the application could link to other notes as an archive. These are all long term goals and I will be focusing on the first part mainly.

The methodology that the project will follow will be an adapted Extreme Programming. There needs to be investigatory work on how this can be applied to this project, but during the development stage then CRC cards could be beneficial when thinking about the design stage.
%==============================================================================
\section{Proposed tasks}
%==============================================================================
The following tasks are ones which should be completed on the project:
\begin{itemize}

\item \textbf{Investigation of how to extract handwriting from images.} This task will involve looking and investigating OCR tools to see how well handwriting data can be interpreted. The tool will need to be trained for the author's handwriting.

\item \textbf{Investigation into server configuration.} The server will need to be configured with running a web application alongside external libraries such as the OCR tool.

\item \textbf{Configuration of local work environment.} Configuration of local machines will closely match that of the server, using the same version control, either git or SVN. Configuration  of a continuous integration tool to aid project development will be required.

\item \textbf{Development}
  \begin{itemize}

    \item \textbf{Produce a front end application to input a note.} The front end features must allow a user to upload an image and see the image on the screen. They can add appropriate tags for the module code and they can search for the module code, producing a full list of notes based on the module code.

    \item \textbf{Back end parsing the images.} The core business logic should conduct basic OCR recognition of text at the top of the notes. The notes can then inteact with OpenCV's libraries to extract diagrams from the notes. Finally, the backend module should integrate with a calendar to archive the notes so they can be found again via the date.

  \end{itemize}

\item \textbf{Produce a list of constraints that the notes must follow.} As notes are varying from person to person, then a constraint should be attached to the notes to ensure the notes follow a similar format. This should be a small set of rules that can be added to the application which represent a formalised structure.

\item \textbf{Investigation of how to extract diagrams and graphs from images.} There will be research conducted to identify an efficient way to identify images, graphs and diagrams from an image of a note. This could then be implemented into the system.

\item \textbf{Continuous project meetings and diaries.} The project will consist of weekly meetings with the supervisor. There shall be one group meeting and another individual meeting to show and share progress. A weekly diary will be kept to monitor individual progress on the project and it will be referenced to in the final report deliverable.

\item \textbf{Preparation for the demonstrations.} The project will consist of a mid project demonstration and an end demonstration. The mid project demonstration will need to be prepared to show minimal character recognition of notes and the ability to upload an image. The final project demonstration should show the ability to search for a note by tagging it and archiving it in a calendar.

\end{itemize}

%==============================================================================
\section{Project deliverables}
%==============================================================================
The following are deliverables which are expected to be completed on the project:
\begin{itemize}

  \item \textbf{The MapMyNotes software.} There should be a web application which at the minimum will take an image and save it to a database and integrate with a calendar. This should be well tested and well structured with appropriate comments where necessary along with providing any build scripts from the server.

  \item \textbf{A collection of user acceptance and model tests.} A series of user acceptance and integration tests should be provided for the front end application. Additionally, back end models should have appropriate logic tested.

  \item \textbf{Story Card, CRC cards, burndown charts and backlog items.} As the author intends to use an Extreme Programming methodology for their project, then providing CRC cards and story cards would be useful for the final report deliverable to show the author's design decisions.

  \item \textbf{OCR training data.} The OCR tool will need to be trained to recognise the author's handwriting. Any training data which aided with this learning stage should be provided.

  \item \textbf{Weekly blogposts regarding progress.} There should be a weekly blog post to aid the author in analysing and reflecting on the week and any obstacles they may have overcome. This will be used and referenced in the final report.

  \item \textbf{Mid-project Demonstration.} There will be a mid project demonstration which should show current progress on the project.

  \item \textbf{Final report} - The report will discuss the work that has been carried out, the process that I have followed any libraries or frameworks which have been used throughout the project. It will discuss the project, the design undertaken and evaluating the end outcome and any changes that would be made.

  \item \textbf{Final Demonstration} - The demonstration has been added as it is a milestone in the application and should be considered when identifying work to complete.

\end{itemize}

%
% Start to comment out / remove the following lines. They are only provided for instruction for this example template.  You don't need the following section title, because it will be added as part of the bibliography section.
%
%==============================================================================
%\section*{Your Bibliography - REMOVE this title and text for final version}
%==============================================================================
%
%You need to include an annotated bibliography. This should list all relevant web pages, books, journals etc. that you have consulted in researching your project. Each reference should include an annotation.

%The purpose of the section is to understand what sources you are looking at.  A correctly formatted list of items and annotations is sufficient. You might go further and make use of bibliographic tools, e.g. BibTeX in a LaTeX document, could be used to provide citations, for example \cite{NumericalRecipes} \cite{MarksPaper} \cite[99-101]{FailBlog} \cite{kittenpic_ref}.  The bibliographic tools are not a requirement, but you are welcome to use them.

%You can remove the above {\em Your Bibliography} section heading because it will be added in by the renewcommand which is part of the bibliography. The correct annotated bibliography information is provided below.
%
% End of comment out / remove the lines. They are only provided for instruction for this example template.
%


\nocite{*} % include everything from the bibliography, irrespective of whether it has been referenced.

% the following line is included so that the bibliography is also shown in the table of contents. There is the possibility that this is added to the previous page for the bibliography. To address this, a newline is added so that it appears on the first page for the bibliography.
\newpage
\addcontentsline{toc}{section}{Initial Annotated Bibliography}

%
% example of including an annotated bibliography. The current style is an author date one. If you want to change, comment out the line and uncomment the subsequent line. You should also modify the packages included at the top (see the notes earlier in the file) and then trash your aux files and re-run.
%\bibliographystyle{authordate2annot}
\bibliographystyle{IEEEannot}
\renewcommand{\refname}{Annotated Bibliography}  % if you put text into the final {} on this line, you will get an extra title, e.g. References. This isn't necessary for the outline project specification.
\bibliography{mmp} % References file

\end{document}
