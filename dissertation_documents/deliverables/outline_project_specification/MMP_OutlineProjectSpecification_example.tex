\documentclass[11pt,fleqn,twoside]{article}
\usepackage{makeidx}
\makeindex
\usepackage{palatino} %or {times} etc
\usepackage{plain} %bibliography style
\usepackage{amsmath} %math fonts - just in case
\usepackage{amsfonts} %math fonts
\usepackage{amssymb} %math fonts
\usepackage{lastpage} %for footer page numbers
\usepackage{fancyhdr} %header and footer package
\usepackage{mmpv2}
\usepackage{url}

% the following packages are used for citations - You only need to include one.
%
% Use the cite package if you are using the numeric style (e.g. IEEEannot).
% Use the natbib package if you are using the author-date style (e.g. authordate2annot).
% Only use one of these and comment out the other one.
\usepackage{cite}
%\usepackage{natbib}

\begin{document}

\name{Ryan Gouldsmith}
\userid{ryg1}
\projecttitle{MapMyNotes}
\projecttitlememoir{MapMyNotes} %same as the project title or abridged version for page header
\reporttitle{Outline Project Specification}
\version{0.3}
\docstatus{Draft}
\modulecode{CS39440}
\degreeschemecode{G401}
\degreeschemename{Computer Science}
\supervisor{Hannah Dee} % e.g. Neil Taylor
\supervisorid{hmd}
\wordcount{}

%optional - comment out next line to use current date for the document
%\documentdate{10th February 2014}
\mmp

\setcounter{tocdepth}{3} %set required number of level in table of contents


%==============================================================================
\section{Project description}
%==============================================================================
My project is called MapMyNotes and it aims to produce a web application which will allow the user to upload an image of their handwritten notes.

Lots of people still handwrite their notes often leading to ridiculous amounts of paper to search through to find their notes. Modern applications have come around such as One Note, which aid this, but people still like to write their notes.

The project aims to produce a web application to aid the digitalisation of their notes. At its very basic level, the user will interact with a web application and upload an image of their handwritten notes. It will use very simple OCR techniques (most likely from the Tesseract OCR engine) to read the title of the notes, who lectured, the module code and the time at a very basic level of the author's handwriting.

The application can be manually tagged by the user. Additionally, it will then store the notes and the metadata into a database, where a user can search for the tags for a module code and find all associated notes for that user. The database choice for the notes has not been decided, but it will need to consider binary data and metadata, so the choices of a NoSQL and SQL database needs to be considered. Once the notes have been collated they will be added to some form of calendar; OAuth with Google Calendar is a possibility but more research with this needs to be conducted.

If there is sufficient time with the project the application will give the option to do full OCR recognition on the text and produce the entire image as text. This will also incorporate extraction of diagrams and graphs from the image and conduct automatic rotation of the note prior to analysing it. Additionally, they will be able to edit the content of their note and the text associated with it. An additional stretch would be to look at analysing any user's handwriting. Finally, you should be able to link to other notes in the archive.

The methodology that the project will follow will be an adapted Extreme Programming. There needs to be investigatory work on how this can be applied to this project, but during the development stage then CRC cards could be beneficial when thinking about the design stage.
%==============================================================================
\section{Proposed tasks}
%==============================================================================
The following tasks are ones which should be completed on the project:
\begin{itemize}

\item \textbf{Investigation of how to extract handwriting from images.} This task will involve looking and investigating OCR tools such as Tesseract to see how well handwriting data can be interpreted. The tool will need to be trained for the authors handwriting so it can interpret it with a high accuracy rate.

\item \textbf{Investigation into server configuration.} Look into how to host a web application with external dependencies such as Tesseract and Open CV libraries. It will also need to investigate how to store images and metadata into the database reliably as well as deciding on an appropriate library to be able to interact with these.

\item \textbf{Configuration of local work environment.} Configuration of the local environment of my machines to be as similar as the server configuration would be needed to reduce the errors when deploying. Will need to set up a private repository with git, most likely with GitHub [CITE].

\item \textbf{Development}
  \begin{itemize}

    \item \textbf{Produce a front end application to input a note.} The front end features must allow a user to upload an image and see the image on the screen. They can add appropriate tags for the module code. The user can then search for the module code, and it will list all the links to the corresponding notes.

    \item \textbf{Back end parsing the images.} The core business logic should do basic OCR recognition of text at the top of the notes. The other logic could interact with OpenCV's libraries to extract the images from the notes. Finally, the backend module should integrate with some kind of calendar to archive the notes in a calendar for someones use.

  \end{itemize}

\item \textbf{Produce a list of constraints that the notes must follow.} As notes are varying from person to person, then a constraint should be attached to the notes to ensure the notes follow a similar format. This should be a small set of rules that can be added to the application which represent a structure for a given upload on the notes.

\item \textbf{Investigation of how to extract diagrams and graphs from images.} There will be research conducted to identify an efficient way to identify images, graphs and diagrams from an image given the user has followed a set of constraints. If they have followed a set of constraints then they should be able to extract an image into the application.

\item \textbf{Continuous project meetings and diaries.} The project will consist of weekly meetings with the supervisor. There shall be one group meeting and another individual meeting to show and share concerns and progress. A weekly diary will be kept to keep track of how the project is progressing and will be referenced to in the end report.

\item \textbf{Preparation for the demonstrations.} The project will consist of a mid project demonstration and an end demonstration. The mid project demonstration will need to be prepared to show minimal character recognition of notes and the ability to upload a note. The final project demonstration should show the ability to search for a note by tagging it and uploading it to a calendar.

\end{itemize}

%==============================================================================
\section{Project deliverables}
%==============================================================================
The following are deliverables which are expected to be completed on the project:
\begin{itemize}

  \item \textbf{The MapMyNotes software.} There should be a web application which at the minimum will take an image and save it to the database and input it into a calendar. This should be well tested and well structured with appropriate comments where necessary.

  \item \textbf{A collection of user acceptance and model tests.} Due to this being a web application then user acceptance and integration tests should be provided for all the front end applications. Additionally, the models which control the logic should also be provided in the testing sections.

  \item \textbf{Story Card, CRC cards, burndown charts and backlog items.} As the author intends to use an Extreme Programming methodology for their project, then providing CRC cards and story cards would be useful for the Final report deliverable to show how they accomplished the design phase. Additionally, burndown charts show progress and how well estimations have got over the 15 week period. Backlog items could also be aided in discussion of what could be achieved if we had more time.

  \item \textbf{A series of training data.} The OCR will need be learned on at least the authors handwriting so training data will need to be provided to show how the OCR tool has been trained to recognise handwriting for the application.

  \item \textbf{Mid-project Demonstration.} There will be a mid project demonstration which should show current progress on the project.

  \item \textbf{Final report} - The report will discuss the work that has been carried out, the process that I have followed any libraries or frameworks which have been used throughout the project. It will discuss the project, the design undertaken and evaluating the end outcome and any changes that would be made.

  \item \textbf{Final Demonstration} - The demonstration has been added as it is a milestone in the application and should be considered when identifying work to complete.

\end{itemize}

%
% Start to comment out / remove the following lines. They are only provided for instruction for this example template.  You don't need the following section title, because it will be added as part of the bibliography section.
%
%==============================================================================
%\section*{Your Bibliography - REMOVE this title and text for final version}
%==============================================================================
%
%You need to include an annotated bibliography. This should list all relevant web pages, books, journals etc. that you have consulted in researching your project. Each reference should include an annotation.

%The purpose of the section is to understand what sources you are looking at.  A correctly formatted list of items and annotations is sufficient. You might go further and make use of bibliographic tools, e.g. BibTeX in a LaTeX document, could be used to provide citations, for example \cite{NumericalRecipes} \cite{MarksPaper} \cite[99-101]{FailBlog} \cite{kittenpic_ref}.  The bibliographic tools are not a requirement, but you are welcome to use them.

%You can remove the above {\em Your Bibliography} section heading because it will be added in by the renewcommand which is part of the bibliography. The correct annotated bibliography information is provided below.
%
% End of comment out / remove the lines. They are only provided for instruction for this example template.
%


\nocite{*} % include everything from the bibliography, irrespective of whether it has been referenced.

% the following line is included so that the bibliography is also shown in the table of contents. There is the possibility that this is added to the previous page for the bibliography. To address this, a newline is added so that it appears on the first page for the bibliography.
\newpage
\addcontentsline{toc}{section}{Initial Annotated Bibliography}

%
% example of including an annotated bibliography. The current style is an author date one. If you want to change, comment out the line and uncomment the subsequent line. You should also modify the packages included at the top (see the notes earlier in the file) and then trash your aux files and re-run.
%\bibliographystyle{authordate2annot}
\bibliographystyle{IEEEannot}
\renewcommand{\refname}{Annotated Bibliography}  % if you put text into the final {} on this line, you will get an extra title, e.g. References. This isn't necessary for the outline project specification.
\bibliography{mmp} % References file

\end{document}
