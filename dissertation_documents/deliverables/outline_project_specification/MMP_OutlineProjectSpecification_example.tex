\documentclass[11pt,fleqn,twoside]{article}
\usepackage{makeidx}
\makeindex
\usepackage{palatino} %or {times} etc
\usepackage{plain} %bibliography style
\usepackage{amsmath} %math fonts - just in case
\usepackage{amsfonts} %math fonts
\usepackage{amssymb} %math fonts
\usepackage{lastpage} %for footer page numbers
\usepackage{fancyhdr} %header and footer package
\usepackage{mmpv2}
\usepackage{url}

% the following packages are used for citations - You only need to include one.
%
% Use the cite package if you are using the numeric style (e.g. IEEEannot).
% Use the natbib package if you are using the author-date style (e.g. authordate2annot).
% Only use one of these and comment out the other one.
\usepackage{cite}
%\usepackage{natbib}

\begin{document}

\name{Ryan Gouldsmith}
\userid{ryg1}
\projecttitle{MapMyNotes}
\projecttitlememoir{MapMyNotes} %same as the project title or abridged version for page header
\reporttitle{Outline Project Specification}
\version{0.1}
\docstatus{Draft}
\modulecode{CS39440}
\degreeschemecode{G401}
\degreeschemename{Computer Science}
\supervisor{Hannah Dee} % e.g. Neil Taylor
\supervisorid{hmd}
\wordcount{}

%optional - comment out next line to use current date for the document
%\documentdate{10th February 2014}
\mmp

\setcounter{tocdepth}{3} %set required number of level in table of contents


%==============================================================================
\section{Project description}
%==============================================================================
My project is called MapMyNotes and it aims to produce a web application piece of software which will allow the user to upload a photograph of their handwritten notes. When the user uploads their image they have two options, to have the ability to have the note be translated fully by OCR or whether they just want to see the image of the notes and only be able to annotate them.

If they see the image of the notes then there will be automatic tagging of the module code via OCR recognition, this will then archive the notes.

If they select to use full OCR then it will automatically convert the entire image into text based on their hand writing. It will then try to identify the module and metadata to perform automatic tagging of the module.

As well as reading textual information the application will look to extract any images, graphs or diagrams from the note. If they select the full OCR extraction then I will be looking to get the image and put it onto the canvas so they can move the text around it accordingly.

Another key feature is that the application must have searchable notes, so it will have to integrate with a structured database. Map my notes may interact with a PostgreSQL database. After this the application will integrate into a calendar so the notes can be saved for a specific day. The calendar has not been considered but I will be researching into using OAuth with Google calendar to allow me this functionality.

The application will have to interact with an OCR (optical character recognition) tool, and one which I have looked at in great depth is the open source tool Tesseract. It's available under the Apache 2 license, and I will be looking to train handwriting using this tool. Additionally for extracting images I will be looking to use the OpenCV tool. This is under a BSD license so is good enough to use in this project.

The application, from initial looking, seems to have 2 core components. A web application interface and a backend parsing of the notes. For the front end of the application I will be looking to write this in HTML5 and Javascript. I have experience of using ReactJS, so maybe this can be incorporated into the application.

The backend application would have to have access to the openCV and tesseract API's, this means that potentially the backend would have to be written in Python. This has not been decided as of yet.

%==============================================================================
\section{Proposed tasks}
%==============================================================================


%==============================================================================
\section{Project deliverables}
%==============================================================================

%
% Start to comment out / remove the following lines. They are only provided for instruction for this example template.  You don't need the following section title, because it will be added as part of the bibliography section.
%
%==============================================================================
\section*{Your Bibliography - REMOVE this title and text for final version}
%==============================================================================
%
You need to include an annotated bibliography. This should list all relevant web pages, books, journals etc. that you have consulted in researching your project. Each reference should include an annotation.

The purpose of the section is to understand what sources you are looking at.  A correctly formatted list of items and annotations is sufficient. You might go further and make use of bibliographic tools, e.g. BibTeX in a LaTeX document, could be used to provide citations, for example \cite{NumericalRecipes} \cite{MarksPaper} \cite[99-101]{FailBlog} \cite{kittenpic_ref}.  The bibliographic tools are not a requirement, but you are welcome to use them.

You can remove the above {\em Your Bibliography} section heading because it will be added in by the renewcommand which is part of the bibliography. The correct annotated bibliography information is provided below.
%
% End of comment out / remove the lines. They are only provided for instruction for this example template.
%


\nocite{*} % include everything from the bibliography, irrespective of whether it has been referenced.

% the following line is included so that the bibliography is also shown in the table of contents. There is the possibility that this is added to the previous page for the bibliography. To address this, a newline is added so that it appears on the first page for the bibliography.
\newpage
\addcontentsline{toc}{section}{Initial Annotated Bibliography}

%
% example of including an annotated bibliography. The current style is an author date one. If you want to change, comment out the line and uncomment the subsequent line. You should also modify the packages included at the top (see the notes earlier in the file) and then trash your aux files and re-run.
%\bibliographystyle{authordate2annot}
\bibliographystyle{IEEEannot}
\renewcommand{\refname}{Annotated Bibliography}  % if you put text into the final {} on this line, you will get an extra title, e.g. References. This isn't necessary for the outline project specification.
\bibliography{mmp} % References file

\end{document}
